\documentclass[twoside]{article}
\usepackage[article]{lambdabook}
\usepackage{lambdatex}

\title{Modello}
\author{Davide Borra}
\date{}
\makeatletter
\let\runauthor\@author
\let\runtitle\@title
\makeatother
\renewcommand{\sectionmark}[1]{\markright{#1}}
\renewcommand{\footrulewidth}{0.4pt}

%\everymath{\displaystyle}

\begin{document}

\lhead{}
\chead{}
\rhead{Indice}
\rfoot{\runauthor}

\begin{titlepage}
    \newgeometry{left=3cm, right=3cm, bottom=2cm, top =3cm} 
    \pagestyle{empty}
    \begin{center}
        \vspace*{\fill}
        \vspace{0.5cm}
        \textbf{\Huge \runtitle}\\\vspace{5mm}
        \textsc{\Large \runauthor}
        \vspace{5cm}
    \end{center}
    \vspace*{\fill}
    v. 1.0\\
    \rule{0.8\linewidth}{0.5mm}\\
    {\footnotesize\href{mailto:davide.borra@studenti.unitn.it}{davide.borra@studenti.unitn.it} - \href{http://davideborra.github.io}{davideborra.github.io}}
    \restoregeometry\newpage
    \thispagestyle{empty}
\end{titlepage}
    \begin{abstract}
        \centering <sommario>
    \end{abstract}
    \tableofcontents
    \creativecommons
    \newpage
    
    \chead{}
    \rfoot{\runauthor}
    \fancyhead[RO]{\nouppercase{\rightmark}}
    \fancyhead[LE]{\nouppercase{\rightmark}}
    \fancyhead[RE]{\runtitle}
    \fancyhead[LO]{\runtitle}
    %%%%%%%%%%%%%%%%%%%%%%%%%%%%%%%%%%%%%%%%%%%%%%%%%%%%%%%%%%%%%%%%%%%%%%%%

    \section{Pippo}


\end{document}